\documentclass[a4paper]{article}
\usepackage[utf8]{inputenc}
\usepackage[T2A]{fontenc}
\setlength{\textheight}{25cm}
\setlength{\textwidth}{18cm}
\setlength{\topmargin}{-25mm}
\setlength{\hoffset}{-25mm}
\def\zn{,\kern-0.09em,}

\begin{document}
\thispagestyle{empty}

\begin{flushleft}
Математички факултет\\
Универзитета у Београду
\end{flushleft}

\bigskip

\begin{center}
\textbf{МОЛБА\\
ЗА ОДОБРАВАЊЕ ТЕМЕ МАСТЕР РАДА
}\end{center}

\bigskip

\begin{flushleft}
Молим да ми се одобри израда мастер рада под насловом:
\end{flushleft}

\begin{minipage}{16.5cm}
%%%%%%%%%%%%%%%%%%%%%%%%%%%%%%%%%%%%%%%%%%%%%%%%%%%%%%%%%%%%%%%%%%%%%%%%%%%%%%%
% U donji red upisati naziv master rada umesto teksta: >>Назив мастер рада<<  %
%%%%%%%%%%%%%%%%%%%%%%%%%%%%%%%%%%%%%%%%%%%%%%%%%%%%%%%%%%%%%%%%%%%%%%%%%%%%%%%
\textbf{{\zn Креирање виџета у програмском језику \textit{Swift}''}}
\end{minipage}\\
\rule[4mm]{17.5cm}{.05mm}
\begin{flushleft}
\framebox{
\begin{minipage}[t][10cm]{17cm}
%%%%%%%%%%%%%%%%%%%%%%%%%%%%%%%%%%%%%%%%%%%%%%%%%%%%%%%%%%%%%%%%%%%%%%%%%%%%%%%
% 	-- unutrasnjost pravougaonika --    	  								  %
%%%%%%%%%%%%%%%%%%%%%%%%%%%%%%%%%%%%%%%%%%%%%%%%%%%%%%%%%%%%%%%%%%%%%%%%%%%%%%%
\textbf{Значај теме и области:}

% 	Umesto donjeg teksta opisati značaj teme i oblasti	%

\textit{Swift} је модеран програмски језик опште намене, настао 2014. године у оквиру компаније \textit{Apple}. Првенствено је намењен развоју апликација на њиховим платформама (\textit{iOS}, \textit{iPadOS}, \textit{macOS}, \textit{tvOS} и \textit{watchOS}), али је део развоја језика усмерен ка поједностављивању процеса израде корисничког интерфејса и увођењу декларативне синтаксе у језик. У том контексту настало је радно окружење \textit{SwiftUI} које одликује могућност брзог креирања концизних и ефикасних решења. Пример примене радног окружења \textit{SwiftUI} је израда виџета намењених оперативним системима \textit{iOS}, \textit{iPadOS} и \textit{macOS}.\\
Виџет, као део мобилне апликације, се налази на почетном екрану уређаја (телефона или таблета) и кориснику приказује одабране важне информације из те апликације. За разлику од виџета у оперативном систему \textit{Android}, који су присутни више од десет година, виџети на \textit{Apple} платформама су уведени 2020. године, тако да је и сама технологија која подржава њихово креирање и даље у активном развоју.

\textbf{Специфични циљ рада:} \\
% 	Umesto donjeg teksta opisati specifični cilj master rada %
Циљ рада је истраживање и опис, карактеристика и могућности које пружа програмски језик \textit{Swift}, промене и побољшања које је донело радно окружење \textit{SwiftUI} и приказ новина насталих увођењем виџета (конкретно у оперативном систему \textit{iOS}). У оквиру мастер рада биће имплементирана \textit{iOS} апликација о кулинарству коришћењем програмског језика \textit{Swift}, након чега ће бити креиран одговарајући виџет помоћу радног окружења \textit{SwiftUI}. Овај виџет пружаће кориснику помоћ приликом избора и припреме оброка.

\textbf{Остале битне информације:}

% 	Umesto donjeg teksta навести друге битне информације %
Литература: \\
1) Програмски језик Swift, https://www.swift.org/ \\
2) The Swift Programming Language (Swift 5.5), Apple Inc. (2014)

% Овај документ је UTF-8 кодиран и неопходно је да тако и остане
% како би се исправно видела ћирилична слова.
\end{minipage}
}
\end{flushleft}
\vspace{1cm}
%%%%%%%%%%%%%%%%%%%%%%%%%%%%%%%%%%%%%%%%%%%%%%%%%%%%%%%%%%%%%%%%%%%%%%%%%%%%%%%
% u donji red uneti:       ime i prezime, broj indeksa i modul studenta       %
%%%%%%%%%%%%%%%%%%%%%%%%%%%%%%%%%%%%%%%%%%%%%%%%%%%%%%%%%%%%%%%%%%%%%%%%%%%%%%%
\makebox[10cm][c]{\textbf{Марко Вељковић, 1096/2019, Информатика}}
%%%%%%%%%%%%%%%%%%%%%%%%%%%%%%%%%%%%%%%%%%%%%%%%%%%%%%%%%%%%%%%%%%%%%%%%%%%%%%%
% u donji red uneti:                   ime i prezime mentora				  %
%%%%%%%%%%%%%%%%%%%%%%%%%%%%%%%%%%%%%%%%%%%%%%%%%%%%%%%%%%%%%%%%%%%%%%%%%%%%%%%
Сагласан ментор \makebox[6cm][c]{\textbf{ др Милена Вујошевић Јаничић}} \\
\rule[4mm]{10cm}{.05mm} \hfill \raisebox{4mm}{\makebox[5cm][l]{.\dotfill.}} \\
\raisebox{1cm}%
[9mm][0mm]{\makebox[10cm][c]{\textit{(име и презиме студента, бр. индекса, модул)}}} \\
\makebox[10cm]{ }\\
\vspace{-1cm}\\
\rule[2cm]{6.5cm}{.05mm} \hfill \rule[2cm]{6.5cm}{.05mm}\\
\vspace{-2.4cm}\\
\raisebox{2cm}{\makebox[6.5cm][c]{\textit{(својеручни потпис студента)}}}
\hfill \raisebox{2cm}{\makebox[6.5cm][c]{\textit{(својеручни потпис ментора)}}}\\
\vspace{-2cm}\\
%%%%%%%%%%%%%%%%%%%%%%%%%%%%%%%%%%%%%%%%%%%%%%%%%%%%%%%%%%%%%%%%%%%%%%%%%%%%%%%
% u donji red uneti datum podnosenja molbe									  %
%%%%%%%%%%%%%%%%%%%%%%%%%%%%%%%%%%%%%%%%%%%%%%%%%%%%%%%%%%%%%%%%%%%%%%%%%%%%%%%
\makebox[5.5cm][c]{\textbf{}}\makebox[5.5cm]{}  Чланови комисије\\
%%%%%%%%%%%%%%%%%%%%%%%%%%%%%%%%%%%%%%%%%%%%%%%%%%%%%%%%%%%%%%%%%%%%%%%%%%%%%%%
% POPUNJAVA MENTOR (rucno ili na sledeci nacin):							  %
% u donji red umesto .\dotfill. upisati podatke o 1. clanu komisije		      %
%%%%%%%%%%%%%%%%%%%%%%%%%%%%%%%%%%%%%%%%%%%%%%%%%%%%%%%%%%%%%%%%%%%%%%%%%%%%%%%
\rule[4mm]{5.5cm}{.05mm}\makebox[5.5cm]{ } 1. \makebox[6cm][l]{.\dotfill.}\\
\vspace{-8mm}\\
\raisebox{4mm}%														
[7mm][0mm]{\makebox[5.5cm][c]{\textit{(датум подношења молбе)}}}\makebox[5.5cm]{ }
%%%%%%%%%%%%%%%%%%%%%%%%%%%%%%%%%%%%%%%%%%%%%%%%%%%%%%%%%%%%%%%%%%%%%%%%%%%%%%%
% POPUNJAVA MENTOR (rucno ili na sledeci nacin): 							  %
% u donji red umesto .\dotfill. upisati podatke o 2. clanu komisije           %
%%%%%%%%%%%%%%%%%%%%%%%%%%%%%%%%%%%%%%%%%%%%%%%%%%%%%%%%%%%%%%%%%%%%%%%%%%%%%%%
2. \makebox[6cm][l]{.\dotfill.}\\

\vspace{1cm}


\begin{flushleft}
%%%%%%%%%%%%%%%%%%%%%%%%%%%%%%%%%%%%%%%%%%%%%%%%%%%%%%%%%%%%%%%%%%%%%%%%%%%%%%%
% u donji red upisati              katedru									  %
%%%%%%%%%%%%%%%%%%%%%%%%%%%%%%%%%%%%%%%%%%%%%%%%%%%%%%%%%%%%%%%%%%%%%%%%%%%%%%%
Катедра \makebox[9.5cm][l]{\textbf{за рачунарство и информатику}} је сагласна са предложеном темом.
\vspace{-3mm}
\hspace*{13mm} \rule[2.3cm]{9.5cm}{.05mm}\\
\vspace{-1cm}
%%%%%%%%%%%%%%%%%%%%%%%%%%%%%%%%%%%%%%%%%%%%%%%%%%%%%%%%%%%%%%%%%%%%%%%%%%%%%%
% POPUNJAVA SEF KATEDRE                                                      %
%%%%%%%%%%%%%%%%%%%%%%%%%%%%%%%%%%%%%%%%%%%%%%%%%%%%%%%%%%%%%%%%%%%%%%%%%%%%%%
\makebox[6.5cm][c]{} \hfill \makebox[6.5cm][c]{}\\
\rule[4mm]{6.5cm}{.05mm} \hfill \rule[4mm]{6.5cm}{.05mm}\\
\vspace{-5mm}
\makebox[6.5cm][c]{\textit{(шеф катедре)}} \hfill \makebox[6.5cm][c]{\textit{(датум одобравања молбе)}}
\end{flushleft}
\end{document} 